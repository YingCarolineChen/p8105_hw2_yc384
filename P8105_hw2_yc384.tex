\documentclass[]{article}
\usepackage{lmodern}
\usepackage{amssymb,amsmath}
\usepackage{ifxetex,ifluatex}
\usepackage{fixltx2e} % provides \textsubscript
\ifnum 0\ifxetex 1\fi\ifluatex 1\fi=0 % if pdftex
  \usepackage[T1]{fontenc}
  \usepackage[utf8]{inputenc}
\else % if luatex or xelatex
  \ifxetex
    \usepackage{mathspec}
  \else
    \usepackage{fontspec}
  \fi
  \defaultfontfeatures{Ligatures=TeX,Scale=MatchLowercase}
\fi
% use upquote if available, for straight quotes in verbatim environments
\IfFileExists{upquote.sty}{\usepackage{upquote}}{}
% use microtype if available
\IfFileExists{microtype.sty}{%
\usepackage{microtype}
\UseMicrotypeSet[protrusion]{basicmath} % disable protrusion for tt fonts
}{}
\usepackage[margin=1in]{geometry}
\usepackage{hyperref}
\hypersetup{unicode=true,
            pdftitle={P8105\_hw2\_yc384},
            pdfauthor={Ying Chen (UNI: yc384)},
            pdfborder={0 0 0},
            breaklinks=true}
\urlstyle{same}  % don't use monospace font for urls
\usepackage{color}
\usepackage{fancyvrb}
\newcommand{\VerbBar}{|}
\newcommand{\VERB}{\Verb[commandchars=\\\{\}]}
\DefineVerbatimEnvironment{Highlighting}{Verbatim}{commandchars=\\\{\}}
% Add ',fontsize=\small' for more characters per line
\usepackage{framed}
\definecolor{shadecolor}{RGB}{248,248,248}
\newenvironment{Shaded}{\begin{snugshade}}{\end{snugshade}}
\newcommand{\AlertTok}[1]{\textcolor[rgb]{0.94,0.16,0.16}{#1}}
\newcommand{\AnnotationTok}[1]{\textcolor[rgb]{0.56,0.35,0.01}{\textbf{\textit{#1}}}}
\newcommand{\AttributeTok}[1]{\textcolor[rgb]{0.77,0.63,0.00}{#1}}
\newcommand{\BaseNTok}[1]{\textcolor[rgb]{0.00,0.00,0.81}{#1}}
\newcommand{\BuiltInTok}[1]{#1}
\newcommand{\CharTok}[1]{\textcolor[rgb]{0.31,0.60,0.02}{#1}}
\newcommand{\CommentTok}[1]{\textcolor[rgb]{0.56,0.35,0.01}{\textit{#1}}}
\newcommand{\CommentVarTok}[1]{\textcolor[rgb]{0.56,0.35,0.01}{\textbf{\textit{#1}}}}
\newcommand{\ConstantTok}[1]{\textcolor[rgb]{0.00,0.00,0.00}{#1}}
\newcommand{\ControlFlowTok}[1]{\textcolor[rgb]{0.13,0.29,0.53}{\textbf{#1}}}
\newcommand{\DataTypeTok}[1]{\textcolor[rgb]{0.13,0.29,0.53}{#1}}
\newcommand{\DecValTok}[1]{\textcolor[rgb]{0.00,0.00,0.81}{#1}}
\newcommand{\DocumentationTok}[1]{\textcolor[rgb]{0.56,0.35,0.01}{\textbf{\textit{#1}}}}
\newcommand{\ErrorTok}[1]{\textcolor[rgb]{0.64,0.00,0.00}{\textbf{#1}}}
\newcommand{\ExtensionTok}[1]{#1}
\newcommand{\FloatTok}[1]{\textcolor[rgb]{0.00,0.00,0.81}{#1}}
\newcommand{\FunctionTok}[1]{\textcolor[rgb]{0.00,0.00,0.00}{#1}}
\newcommand{\ImportTok}[1]{#1}
\newcommand{\InformationTok}[1]{\textcolor[rgb]{0.56,0.35,0.01}{\textbf{\textit{#1}}}}
\newcommand{\KeywordTok}[1]{\textcolor[rgb]{0.13,0.29,0.53}{\textbf{#1}}}
\newcommand{\NormalTok}[1]{#1}
\newcommand{\OperatorTok}[1]{\textcolor[rgb]{0.81,0.36,0.00}{\textbf{#1}}}
\newcommand{\OtherTok}[1]{\textcolor[rgb]{0.56,0.35,0.01}{#1}}
\newcommand{\PreprocessorTok}[1]{\textcolor[rgb]{0.56,0.35,0.01}{\textit{#1}}}
\newcommand{\RegionMarkerTok}[1]{#1}
\newcommand{\SpecialCharTok}[1]{\textcolor[rgb]{0.00,0.00,0.00}{#1}}
\newcommand{\SpecialStringTok}[1]{\textcolor[rgb]{0.31,0.60,0.02}{#1}}
\newcommand{\StringTok}[1]{\textcolor[rgb]{0.31,0.60,0.02}{#1}}
\newcommand{\VariableTok}[1]{\textcolor[rgb]{0.00,0.00,0.00}{#1}}
\newcommand{\VerbatimStringTok}[1]{\textcolor[rgb]{0.31,0.60,0.02}{#1}}
\newcommand{\WarningTok}[1]{\textcolor[rgb]{0.56,0.35,0.01}{\textbf{\textit{#1}}}}
\usepackage{graphicx,grffile}
\makeatletter
\def\maxwidth{\ifdim\Gin@nat@width>\linewidth\linewidth\else\Gin@nat@width\fi}
\def\maxheight{\ifdim\Gin@nat@height>\textheight\textheight\else\Gin@nat@height\fi}
\makeatother
% Scale images if necessary, so that they will not overflow the page
% margins by default, and it is still possible to overwrite the defaults
% using explicit options in \includegraphics[width, height, ...]{}
\setkeys{Gin}{width=\maxwidth,height=\maxheight,keepaspectratio}
\IfFileExists{parskip.sty}{%
\usepackage{parskip}
}{% else
\setlength{\parindent}{0pt}
\setlength{\parskip}{6pt plus 2pt minus 1pt}
}
\setlength{\emergencystretch}{3em}  % prevent overfull lines
\providecommand{\tightlist}{%
  \setlength{\itemsep}{0pt}\setlength{\parskip}{0pt}}
\setcounter{secnumdepth}{0}
% Redefines (sub)paragraphs to behave more like sections
\ifx\paragraph\undefined\else
\let\oldparagraph\paragraph
\renewcommand{\paragraph}[1]{\oldparagraph{#1}\mbox{}}
\fi
\ifx\subparagraph\undefined\else
\let\oldsubparagraph\subparagraph
\renewcommand{\subparagraph}[1]{\oldsubparagraph{#1}\mbox{}}
\fi

%%% Use protect on footnotes to avoid problems with footnotes in titles
\let\rmarkdownfootnote\footnote%
\def\footnote{\protect\rmarkdownfootnote}

%%% Change title format to be more compact
\usepackage{titling}

% Create subtitle command for use in maketitle
\providecommand{\subtitle}[1]{
  \posttitle{
    \begin{center}\large#1\end{center}
    }
}

\setlength{\droptitle}{-2em}

  \title{P8105\_hw2\_yc384}
    \pretitle{\vspace{\droptitle}\centering\huge}
  \posttitle{\par}
    \author{Ying Chen (UNI: yc384)}
    \preauthor{\centering\large\emph}
  \postauthor{\par}
      \predate{\centering\large\emph}
  \postdate{\par}
    \date{9/24/2019}


\begin{document}
\maketitle

setwd(``/Users/macbook/Documents/Statistics/P8105/HW/P8105 HW2'')

\begin{Shaded}
\begin{Highlighting}[]
\NormalTok{knitr}\OperatorTok{::}\NormalTok{opts_chunk}\OperatorTok{$}\KeywordTok{set}\NormalTok{(}\DataTypeTok{echo =} \OtherTok{TRUE}\NormalTok{)}
\KeywordTok{library}\NormalTok{ (tidyverse)}
\end{Highlighting}
\end{Shaded}

\begin{verbatim}
## -- Attaching packages ---------------------------------------------------------------------------------------------- tidyverse 1.2.1 --
\end{verbatim}

\begin{verbatim}
## v ggplot2 3.2.1     v purrr   0.3.2
## v tibble  2.1.3     v dplyr   0.8.3
## v tidyr   1.0.0     v stringr 1.4.0
## v readr   1.3.1     v forcats 0.4.0
\end{verbatim}

\begin{verbatim}
## -- Conflicts ------------------------------------------------------------------------------------------------- tidyverse_conflicts() --
## x dplyr::filter() masks stats::filter()
## x dplyr::lag()    masks stats::lag()
\end{verbatim}

\begin{Shaded}
\begin{Highlighting}[]
\KeywordTok{library}\NormalTok{ (dplyr)}
\KeywordTok{library}\NormalTok{(readxl)}
\KeywordTok{rm}\NormalTok{(}\DataTypeTok{list =} \KeywordTok{ls}\NormalTok{())}

\KeywordTok{options}\NormalTok{(}\DataTypeTok{tibble.print_min =} \DecValTok{3}\NormalTok{)}
\end{Highlighting}
\end{Shaded}

\hypertarget{p8105-dsi-homework-1}{%
\subsection{P8105 DSI Homework 1}\label{p8105-dsi-homework-1}}

\hypertarget{this-assignment-reinforces-ideas-in-data-wrangling-i}{%
\subsubsection{This assignment reinforces ideas in Data Wrangling
I}\label{this-assignment-reinforces-ideas-in-data-wrangling-i}}

\hypertarget{problem-0}{%
\paragraph{1. Problem 0}\label{problem-0}}

\begin{verbatim}
  * Github repo: https://github.com/YingCarolineChen/p8105_hw2_yc384.git
  * RMarkdown file name: P8105_hw2_yc384
  * Create a subdirectory to store local data files and the path is:
    "~/Documents/Statistics/P8105/HW/P8105 HW2"
\end{verbatim}

\hypertarget{problem-1}{%
\subsubsection{2. Problem 1}\label{problem-1}}

\begin{verbatim}
  * 1-1 Read and tidy Mr. Trash Wheel data
  * Dataset: Mr. Trash Wheel / MS Excel file / Contains 8 sheets
\end{verbatim}

\begin{Shaded}
\begin{Highlighting}[]
\CommentTok{# read in Mr. Trash Wheel sheet}
\CommentTok{# skip first rows with notes / figures}
\CommentTok{# drop last column that containing notes}
\NormalTok{MrTrashWheel =}\StringTok{ }
\StringTok{  }\KeywordTok{read_excel}\NormalTok{(}\StringTok{"./data/HealthyHarborWaterWheelTotals2018-7-28.xlsx"}\NormalTok{, }\DataTypeTok{sheet =} \StringTok{"Mr. Trash Wheel"}\NormalTok{, }\DataTypeTok{skip =} \DecValTok{1}\NormalTok{, }\DataTypeTok{range =} \StringTok{"A2:N338"}\NormalTok{, }\DataTypeTok{col_names =} \OtherTok{TRUE}\NormalTok{) }\OperatorTok\StringTok{ }
\StringTok{  }\NormalTok{janitor}\OperatorTok{::}\KeywordTok{clean_names}\NormalTok{()}

\CommentTok{# omit rows has NA for dumpster}
\NormalTok{dumpster =}\StringTok{ }
\StringTok{  }\KeywordTok{drop_na}\NormalTok{(MrTrashWheel,dumpster) }\OperatorTok\StringTok{ }
\StringTok{  }\CommentTok{# Round sports balls to nearest integer}
\StringTok{  }\KeywordTok{mutate}\NormalTok{(}
  \DataTypeTok{sports_balls_round =} \KeywordTok{round}\NormalTok{(sports_balls, }\DataTypeTok{digits =} \DecValTok{0}\NormalTok{),}
  \DataTypeTok{sports_balls_int =} \KeywordTok{as.integer}\NormalTok{(sports_balls_round)}
\NormalTok{)}

\CommentTok{# rename to reasonable variable names}
\NormalTok{dumpster_rename =}\StringTok{ }\KeywordTok{rename}\NormalTok{(dumpster, }\DataTypeTok{bags_grocery =}\NormalTok{ grocery_bags, }\DataTypeTok{bags_chip  =}\NormalTok{ chip_bags, }\DataTypeTok{bottles_plastic =}\NormalTok{ plastic_bottles, }\DataTypeTok{bottles_glass =}\NormalTok{ glass_bottles)}

\CommentTok{# Check data}
\KeywordTok{head}\NormalTok{(MrTrashWheel,}\DecValTok{5}\NormalTok{)}
\end{Highlighting}
\end{Shaded}

\begin{verbatim}
## # A tibble: 5 x 14
##   dumpster month  year date                weight_tons volume_cubic_ya~
##      <dbl> <chr> <dbl> <dttm>                    <dbl>            <dbl>
## 1        1 May    2014 2014-05-16 00:00:00        4.31               18
## 2        2 May    2014 2014-05-16 00:00:00        2.74               13
## 3        3 May    2014 2014-05-16 00:00:00        3.45               15
## 4        4 May    2014 2014-05-17 00:00:00        3.1                15
## 5        5 May    2014 2014-05-17 00:00:00        4.06               18
## # ... with 8 more variables: plastic_bottles <dbl>, polystyrene <dbl>,
## #   cigarette_butts <dbl>, glass_bottles <dbl>, grocery_bags <dbl>,
## #   chip_bags <dbl>, sports_balls <dbl>, homes_powered <dbl>
\end{verbatim}

\begin{Shaded}
\begin{Highlighting}[]
\KeywordTok{head}\NormalTok{(dumpster_rename, }\DecValTok{5}\NormalTok{)}
\end{Highlighting}
\end{Shaded}

\begin{verbatim}
## # A tibble: 5 x 16
##   dumpster month  year date                weight_tons volume_cubic_ya~
##      <dbl> <chr> <dbl> <dttm>                    <dbl>            <dbl>
## 1        1 May    2014 2014-05-16 00:00:00        4.31               18
## 2        2 May    2014 2014-05-16 00:00:00        2.74               13
## 3        3 May    2014 2014-05-16 00:00:00        3.45               15
## 4        4 May    2014 2014-05-17 00:00:00        3.1                15
## 5        5 May    2014 2014-05-17 00:00:00        4.06               18
## # ... with 10 more variables: bottles_plastic <dbl>, polystyrene <dbl>,
## #   cigarette_butts <dbl>, bottles_glass <dbl>, bags_grocery <dbl>,
## #   bags_chip <dbl>, sports_balls <dbl>, homes_powered <dbl>,
## #   sports_balls_round <dbl>, sports_balls_int <int>
\end{verbatim}

\hypertarget{problem-1-1}{%
\subsubsection{2. Problem 1}\label{problem-1-1}}

\begin{verbatim}
  * 1-2 Read and tidy precipitation data
  *     join datasets and some data manipulations
\end{verbatim}

\begin{Shaded}
\begin{Highlighting}[]
\CommentTok{# read in precipitation data for 2017 and 2018}
\CommentTok{# Omit rows without precipitation data and add a variable year}
\NormalTok{precipitation_}\DecValTok{2017}\NormalTok{ =}\StringTok{ }
\StringTok{  }\KeywordTok{read_excel}\NormalTok{(}\StringTok{"./data/HealthyHarborWaterWheelTotals2018-7-28.xlsx"}\NormalTok{, }\DataTypeTok{sheet =} \StringTok{"2017 Precipitation"}\NormalTok{, }\DataTypeTok{skip =} \DecValTok{1}\NormalTok{, }\DataTypeTok{range =} \StringTok{"A2:B14"}\NormalTok{, }\DataTypeTok{col_names =} \OtherTok{TRUE}\NormalTok{) }\OperatorTok\StringTok{ }
\StringTok{  }\NormalTok{janitor}\OperatorTok{::}\KeywordTok{clean_names}\NormalTok{() }\OperatorTok\StringTok{ }
\StringTok{  }\KeywordTok{mutate}\NormalTok{(}
    \DataTypeTok{year =} \DecValTok{2017}\NormalTok{) }\OperatorTok\StringTok{ }
\StringTok{  }\KeywordTok{drop_na}\NormalTok{ (total)}

\NormalTok{precipitation_}\DecValTok{2018}\NormalTok{ =}\StringTok{ }
\StringTok{  }\KeywordTok{read_excel}\NormalTok{(}\StringTok{"./data/HealthyHarborWaterWheelTotals2018-7-28.xlsx"}\NormalTok{, }\DataTypeTok{sheet =} \StringTok{"2018 Precipitation"}\NormalTok{, }\DataTypeTok{skip =} \DecValTok{1}\NormalTok{, }\DataTypeTok{range =} \StringTok{"A2:B14"}\NormalTok{, }\DataTypeTok{col_names =} \OtherTok{TRUE}\NormalTok{) }\OperatorTok\StringTok{ }
\StringTok{  }\NormalTok{janitor}\OperatorTok{::}\KeywordTok{clean_names}\NormalTok{() }\OperatorTok\StringTok{ }
\StringTok{  }\KeywordTok{mutate}\NormalTok{(}
    \DataTypeTok{year =} \DecValTok{2018}\NormalTok{) }\OperatorTok\StringTok{ }
\StringTok{  }\KeywordTok{drop_na}\NormalTok{ (total)}

\CommentTok{# Combine precipitation datasets}
\NormalTok{precipitation_join =}\StringTok{ }
\StringTok{  }\KeywordTok{full_join}\NormalTok{(precipitation_}\DecValTok{2017}\NormalTok{, precipitation_}\DecValTok{2018}\NormalTok{, }\DataTypeTok{by =} \OtherTok{NULL}\NormalTok{) }\OperatorTok\StringTok{ }
\StringTok{  }\CommentTok{# convert month to a character variable}
\StringTok{  }\KeywordTok{mutate}\NormalTok{(}
    \DataTypeTok{month =}\NormalTok{ month.name[month]}
\NormalTok{  )}
\end{Highlighting}
\end{Shaded}

\begin{verbatim}
## Joining, by = c("month", "total", "year")
\end{verbatim}

\begin{Shaded}
\begin{Highlighting}[]
\KeywordTok{head}\NormalTok{(precipitation_join,}\DecValTok{5}\NormalTok{)}
\end{Highlighting}
\end{Shaded}

\begin{verbatim}
## # A tibble: 5 x 3
##   month    total  year
##   <chr>    <dbl> <dbl>
## 1 January   2.34  2017
## 2 February  1.46  2017
## 3 March     3.57  2017
## 4 April     3.99  2017
## 5 May       5.64  2017
\end{verbatim}

\hypertarget{problem-1-2}{%
\subsubsection{2. Problem 1}\label{problem-1-2}}

\hypertarget{datasets-description-using-inline-r-code}{%
\paragraph{* 1-3 Datasets description using inline R
code}\label{datasets-description-using-inline-r-code}}

\begin{Shaded}
\begin{Highlighting}[]
\NormalTok{dump2017 =}\StringTok{ }\KeywordTok{filter}\NormalTok{(dumpster, year }\OperatorTok{==}\StringTok{ }\DecValTok{2017}\NormalTok{)}
\end{Highlighting}
\end{Shaded}

Dataset Mr.~Trash Wheel contains 336 obersvations and 14 variables. The
key variable for this dataset is called ``dumpster''. Variable
``dumpster'' has 51 rows with ``NA'' and will be excluded for future
analyses. The median number of sports balls in a dumpster in 2017 is: 8.

Dataset precipitation\_2017 contains 12 obersvations and 3 variables.
The key variables for this dataset is year and month. Dataset
precipitation\_2018 contains 7 obersvations and 3 variables. The key
variables for this dataset are year and month. Dataset
precipitation\_join is the combination of datasets precipitation\_2017
and precipitation\_2018. The total precipitation is 2018 is 23.5in.

\hypertarget{problem-2}{%
\subsubsection{3. Problem 2}\label{problem-2}}

\hypertarget{read-pols-month.csv-unemployment.csv-and-snp.csv-and-merge-them-using-year-and-month-as-keys}{%
\paragraph{* Read pols-month.csv, unemployment.csv, and snp.csv and
merge them using year and month as
keys}\label{read-pols-month.csv-unemployment.csv-and-snp.csv-and-merge-them-using-year-and-month-as-keys}}

\hypertarget{this-code-cunk-will-read-and-tidy-pols-month.csv.}{%
\paragraph{* 2-1 This code cunk will read and tidy
pols-month.csv.}\label{this-code-cunk-will-read-and-tidy-pols-month.csv.}}

\hypertarget{break-up-the-variable-mon-into-integer-variables-year-month-and-day}{%
\paragraph{* break up the variable mon into integer variables year,
month, and
day;}\label{break-up-the-variable-mon-into-integer-variables-year-month-and-day}}

\hypertarget{replace-month-number-with-month-name}{%
\paragraph{* replace month number with month
name;}\label{replace-month-number-with-month-name}}

\hypertarget{create-a-president-variable-taking-values-gop-and-dem-remove-prez_dem-prez_gop-and-the-day-variable.}{%
\paragraph{* create a president variable taking values gop and dem,
remove prez\_dem, prez\_gop and the day
variable.}\label{create-a-president-variable-taking-values-gop-and-dem-remove-prez_dem-prez_gop-and-the-day-variable.}}

\begin{Shaded}
\begin{Highlighting}[]
\CommentTok{# read in pols-month.csv}
\NormalTok{pols_tidy =}\StringTok{ }
\StringTok{  }\KeywordTok{read_csv}\NormalTok{(}\StringTok{"./data/pols-month.csv"}\NormalTok{, }\DataTypeTok{col_names =} \OtherTok{TRUE}\NormalTok{) }\OperatorTok\StringTok{ }
\StringTok{  }\KeywordTok{mutate}\NormalTok{ (}
    \CommentTok{# change date format}
    \DataTypeTok{Date =} \KeywordTok{as.Date}\NormalTok{(mon, }\DataTypeTok{format=}\StringTok{'%m/%d/%Y'}\NormalTok{)}
\NormalTok{  ) }\OperatorTok\StringTok{ }
\StringTok{  }\NormalTok{janitor}\OperatorTok{::}\KeywordTok{clean_names}\NormalTok{() }\OperatorTok\StringTok{ }
\StringTok{  }\CommentTok{# break date to three vars}
\StringTok{  }\KeywordTok{separate}\NormalTok{(., }\StringTok{"date"}\NormalTok{, }\KeywordTok{c}\NormalTok{(}\StringTok{"year"}\NormalTok{, }\StringTok{"month"}\NormalTok{, }\StringTok{"day"}\NormalTok{)) }\OperatorTok\StringTok{ }
\StringTok{  }\KeywordTok{mutate}\NormalTok{ (}
   \DataTypeTok{month =} \KeywordTok{as.numeric}\NormalTok{(month),}
   \CommentTok{#replace month number with month name}
   \DataTypeTok{month =}\NormalTok{ month.name[}\KeywordTok{c}\NormalTok{(month)],}
   \DataTypeTok{president =}\NormalTok{ prez_gop }\OperatorTok{+}\StringTok{ }\NormalTok{prez_dem}
\NormalTok{  ) }\OperatorTok\StringTok{ }
\StringTok{  }\CommentTok{# reorder varaibles and remove three variables}
\StringTok{  }\KeywordTok{select}\NormalTok{(year, month, }\KeywordTok{everything}\NormalTok{(), }\OperatorTok{-}\NormalTok{prez_dem, }\OperatorTok{-}\NormalTok{prez_gop, }\OperatorTok{-}\NormalTok{day)}
\end{Highlighting}
\end{Shaded}

\begin{verbatim}
## Parsed with column specification:
## cols(
##   mon = col_date(format = ""),
##   prez_gop = col_double(),
##   gov_gop = col_double(),
##   sen_gop = col_double(),
##   rep_gop = col_double(),
##   prez_dem = col_double(),
##   gov_dem = col_double(),
##   sen_dem = col_double(),
##   rep_dem = col_double()
## )
\end{verbatim}

\begin{Shaded}
\begin{Highlighting}[]
\CommentTok{# read in snp.csv}
\NormalTok{snp_tidy =}\StringTok{ }
\StringTok{  }\KeywordTok{read_csv}\NormalTok{(}\StringTok{"./data/snp.csv"}\NormalTok{, }\DataTypeTok{col_names =} \OtherTok{TRUE}\NormalTok{) }\OperatorTok\StringTok{ }
\StringTok{  }\KeywordTok{mutate}\NormalTok{ (}
    \CommentTok{# change date format}
    \DataTypeTok{Date1 =} \KeywordTok{as.Date}\NormalTok{(date, }\DataTypeTok{format=}\StringTok{'%m/%d/%Y'}\NormalTok{)}
\NormalTok{  ) }\OperatorTok\StringTok{ }
\StringTok{  }\NormalTok{janitor}\OperatorTok{::}\KeywordTok{clean_names}\NormalTok{() }\OperatorTok\StringTok{ }
\StringTok{  }\CommentTok{# break date to three vars}
\StringTok{  }\KeywordTok{separate}\NormalTok{(., }\StringTok{"date1"}\NormalTok{, }\KeywordTok{c}\NormalTok{(}\StringTok{"year"}\NormalTok{, }\StringTok{"month"}\NormalTok{, }\StringTok{"day"}\NormalTok{)) }\OperatorTok\StringTok{ }
\StringTok{  }\KeywordTok{mutate}\NormalTok{ (}
   \DataTypeTok{month =} \KeywordTok{as.numeric}\NormalTok{(month),}
   \CommentTok{#replace month number with month name}
   \DataTypeTok{month =}\NormalTok{ month.name[}\KeywordTok{c}\NormalTok{(month)],}
\NormalTok{  ) }\OperatorTok\StringTok{ }
\StringTok{  }\CommentTok{# reorder varaibles and remove three variables}
\StringTok{  }\KeywordTok{select}\NormalTok{(year, month, }\KeywordTok{everything}\NormalTok{(), }\OperatorTok{-}\NormalTok{day)}
\end{Highlighting}
\end{Shaded}

\begin{verbatim}
## Parsed with column specification:
## cols(
##   date = col_character(),
##   close = col_double()
## )
\end{verbatim}

\begin{Shaded}
\begin{Highlighting}[]
\CommentTok{# read in unemployment.csv}
\NormalTok{unemployment =}\StringTok{ }
\StringTok{  }\KeywordTok{read_csv}\NormalTok{(}\StringTok{"./data/unemployment.csv"}\NormalTok{, }\DataTypeTok{col_names =} \OtherTok{TRUE}\NormalTok{) }\OperatorTok\StringTok{ }
\StringTok{  }\NormalTok{janitor}\OperatorTok{::}\KeywordTok{clean_names}\NormalTok{()}
\end{Highlighting}
\end{Shaded}

\begin{verbatim}
## Parsed with column specification:
## cols(
##   Year = col_double(),
##   Jan = col_double(),
##   Feb = col_double(),
##   Mar = col_double(),
##   Apr = col_double(),
##   May = col_double(),
##   Jun = col_double(),
##   Jul = col_double(),
##   Aug = col_double(),
##   Sep = col_double(),
##   Oct = col_double(),
##   Nov = col_double(),
##   Dec = col_double()
## )
\end{verbatim}

\begin{Shaded}
\begin{Highlighting}[]
\NormalTok{unemployment}
\end{Highlighting}
\end{Shaded}

\begin{verbatim}
## # A tibble: 68 x 13
##    year   jan   feb   mar   apr   may   jun   jul   aug   sep   oct   nov
##   <dbl> <dbl> <dbl> <dbl> <dbl> <dbl> <dbl> <dbl> <dbl> <dbl> <dbl> <dbl>
## 1  1948   3.4   3.8   4     3.9   3.5   3.6   3.6   3.9   3.8   3.7   3.8
## 2  1949   4.3   4.7   5     5.3   6.1   6.2   6.7   6.8   6.6   7.9   6.4
## 3  1950   6.5   6.4   6.3   5.8   5.5   5.4   5     4.5   4.4   4.2   4.2
## # ... with 65 more rows, and 1 more variable: dec <dbl>
\end{verbatim}

\begin{Shaded}
\begin{Highlighting}[]
\NormalTok{unemployment_tidy =}\StringTok{ }
\StringTok{  }\KeywordTok{pivot_longer}\NormalTok{(}
\NormalTok{    unemployment,}
\NormalTok{    jan}\OperatorTok{:}\NormalTok{dec,}
    \DataTypeTok{names_to =} \StringTok{"month"}\NormalTok{,}
    \DataTypeTok{values_to =} \StringTok{"unemploy_rate"}
\NormalTok{  )}

\NormalTok{unemployment_tidy}
\end{Highlighting}
\end{Shaded}

\begin{verbatim}
## # A tibble: 816 x 3
##    year month unemploy_rate
##   <dbl> <chr>         <dbl>
## 1  1948 jan             3.4
## 2  1948 feb             3.8
## 3  1948 mar             4  
## # ... with 813 more rows
\end{verbatim}


\end{document}
